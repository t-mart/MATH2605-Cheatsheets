\documentclass[10pt,letterpaper]{article}

\usepackage[landscape,margin=0.1in]{geometry}
\usepackage{multicol}
\usepackage{setspace}
\usepackage[small,compact]{titlesec}
\usepackage{amsmath}
\usepackage{xfrac}

%\setlength{\abovedisplayshortskip}{0pt}
%\setlength{\belowdisplayshortskip}{0pt}
%\setlength{\abovedisplayskip}{0pt}
%\setlength{\belowdisplayskip}{0pt}

\newenvironment{tight_item}
{\begin{itemize}
\setlength{\parskip}{0pt}
\setlength{\parsep}{0pt}
\setlength{\itemsep}{0pt}
\setlength{\parsep}{0pt}
\setlength{\topsep}{0pt}
\setlength{\partopsep}{0pt}
\setlength{\leftmargin}{0em}
\setlength{\labelwidth}{0em}
\setlength{\labelsep}{1em} }
{\end{itemize}}

\newenvironment{tight_enum}
{\begin{enumerate}
\setlength{\parskip}{0pt}
\setlength{\parsep}{0pt}
\setlength{\itemsep}{0pt}
\setlength{\parsep}{0pt}
\setlength{\topsep}{0pt}
\setlength{\partopsep}{0pt}
\setlength{\leftmargin}{0em}
\setlength{\labelwidth}{0em}
\setlength{\labelsep}{1em} }
{\end{enumerate}}

\newenvironment{tight_desc}
{\begin{description}
\setlength{\parskip}{0pt}
\setlength{\parsep}{0pt}
\setlength{\itemsep}{0pt}
\setlength{\parsep}{0pt}
\setlength{\topsep}{0pt}
\setlength{\partopsep}{0pt}
\setlength{\leftmargin}{0em}
\setlength{\labelwidth}{0em}
\setlength{\labelsep}{1em} }
{\end{description}}

\begin{document}

\begin{multicols*}{2}

\section{Iterative methods to solve $A\mathbf{x} = \mathbf{b}$}
In the proceeding methods, start the algorithm with the following:
\begin{tight_enum}
\item \texttt{A = <A>}
\item \texttt{b = <b>}
\item \texttt{x = \{ \{0\}, \{0\} \} }
\end{tight_enum}
Then enter algorithm code. Repeat algorithm code to make an iteration.

\subsection{Jacobi}
\begin{tight_item}
\item $x^{n} = -D^{-1}(U + L)x^{n-1}+ D^{-1}b$
\item $\operatorname{CR} = -D^{-1}(U+L)$
\end{tight_item}
\begin{tight_desc}
\item[Algorithm Code] \texttt{x = (-Inverse[DiagonalMatrix[Diagonal[A]]] .\\(UpperTriangularize[A, 1] + LowerTriangularize[A, -1]) . N[x]) +\\(Inverse[DiagonalMatrix[Diagonal[A]]] . b)}
\item[CR Code] \texttt{Max[Abs[Eigenvalues[-Inverse[DiagonalMatrix[Diagonal[A]]] .\\(UpperTriangularize[A, 1] + LowerTriangularize[A, -1])]]]}
\end{tight_desc}

\subsection{Gauss-Seidel}
\begin{tight_item}
\item $x^{n} = (L+D)^{-1}(b-Ux^{n-1})$
\item $\operatorname{CR} = -(L+D)^{-1}U$
\end{tight_item}
\begin{tight_desc}
\item[Algorithm Code] \texttt{x = (Inverse[LowerTriangularize[A, -1] +\\DiagonalMatrix[Diagonal[A]]]) .\\(b - (UpperTriangularize[A, 1] . N[x]))}
\item[CR Code] \texttt{Max[Abs[Eigenvalues[ -Inverse[L + D] . U]]]}
\end{tight_desc}

\subsection{Convergence Rate (CR)}
\begin{tight_item}
\item Measures rate of convergence per iteration of algorithm towards true values
\item The greater the CR, the quicker the convergence
\end{tight_item}
\begin{tight_enum}
\item Find $\operatorname{CR} = \max{|\lambda_{i}|}$ for algorithm's convergence matrix
\end{tight_enum}

\section{Iterative methods to find eigenvalues}
In the proceeding methods, start the algorithm with the following:
\begin{tight_enum}
\item \texttt{A = <A>}
\end{tight_enum}
Then enter algorithm code. Repeat algorithm code to make an iteration. After a
certain iteration, you can check the eigenvalue with the eigenvalue code.

\subsection{Power Method for Dominant Eigenvalue}
\begin{tight_item}
\item $\mathbf{u}^{n} = \frac{A\textbf{u}^{n-1}}{\|A\textbf{u}^{n-1}\|}$
\item $ \mathbf{\lambda}_{\operatorname{dominant}} = \|A\mathbf{u} \| $
\item "Dominant" means $\max{ | \lambda_{i} | }, \lambda_{i} \in \lambda$
\end{tight_item}
\begin{tight_enum}
\item Assign \texttt{u = \{ \{1\}, \{1\}, \ldots, \{1\} \}}
\item Proceed with algorithm code
\end{tight_enum}
\begin{tight_desc}
\item[Algorithm Code]\texttt{u = N[(A.u)/Norm[A.u]]}
\item[Eigenvalue Code] \texttt{Norm[A.u]} is the dominant eigenvalue
\end{tight_desc}


\subsection{QR Decomposition for Eigenvalues}
\begin{tight_item}
\item $A_{n} = R_{n-1} Q_{n-1}$
\item $ \mathbf{\lambda} = \operatorname{diagonal}(A_{n})$
\end{tight_item}
\begin{tight_enum}
\item Start with a standard QR Decomposition by \texttt{\{q, r\} = QRDecomposition[A]}
\item Proceed with algorithm code
\end{tight_enum}
\begin{tight_desc}
\item[Algorithm Code]\texttt{\{q, r\} = N[QRDecomposition[r . Transpose[q]]]}
\item[Eigenvalue Code] \texttt{Diagonal[Transpose[q] . r]} are the eigenvalues
\end{tight_desc}

\end{multicols*}
\end{document}
